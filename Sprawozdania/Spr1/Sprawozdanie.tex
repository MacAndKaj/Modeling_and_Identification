\documentclass[12pt,a4paper]{article}
\usepackage{amsmath}
\usepackage{amsfonts}
\usepackage{amssymb}
\usepackage{graphicx}
\usepackage[left=2cm,right=2cm,top=2cm,bottom=2cm]{geometry}
\usepackage[T1]{fontenc}
\usepackage[polish]{babel}
\usepackage[utf8]{inputenc}
\usepackage{textcomp}
\usepackage{hyperref}
\usepackage{float}
\usepackage{url}
\usepackage{subcaption}
\usepackage{cite}
\author{Maciej Kajdak \\ Nr indeksu: 226256}
\title{Sprawozdanie 1 \\ \textbf{Modelowanie i Identyfikacja}}
\renewcommand{\thesection}{\arabic{section}}

\begin{document}
\maketitle
\section{Wstęp}
Przy generowaniu liczb losowych w najprostszym przypadku można posłużyć się tzw. generatorami fizycznymi. Do nich można zaliczyć chociażby rzut monetą (losowanie liczb ze zbiory ${0,1}$) lub pobieranie co pewien kwant czasu próbki z przebiegu pewnego szumu elementu elektronicznego. 
Takie rozwiązania są jednak pod wieloma względami niepraktyczne. Postęp, któremu z biegiem lat ulega nauka i technologia pozwala na coraz coraz wydajniejsze programowe generowanie liczb losowych.
Do celu generowania liczb można wykorzystać np.\ generatory liczb losowych o rozkładzie równomiernym. 
Wśród nich najczęściej spotykane są generatory liniowe, nieliniowe, Fibonacciego lub generatory oparte na rejestrach przesuwnych.
Chcąc wygenerować liczby losowe o dowolnym rozkładzie prawdopodobieństwa również można posłużyć się algorytmami powszechnie znanymi, takimi jak np.\ metoda odwracania dystrybuanty albo metoda eliminacji.
Ważnym aspektem generacji liczb losowych jest także testowanie generatorów.
Dzięki temu możliwe jest ocenianie jakości generatora pod konkretnym kątem.
W niniejszym sprawozdaniu opisane zostały wyniki badań, którym zostały poddane różne algorytmy generowania liczb losowych.
Generatory implementowano w środowisku Python w wersji 3.7.1 \cite{python3} wraz z pakietami numpy w wersji 1.15.4 \cite{numpy} oraz matplotlib w wersji 3.0.2 \cite{matplotlib}.
Wszystkie stworzone w ramach laboratoriów skrypty można znaleźć w \href{https://github.com/MacAndKaj/Modeling\_and\_Identification}{\underline{repozytorium twórcy w serwisie Github}}

\section{Generator liczb losowych o rozkładzie jednostajnym}
Jednostajny (równomierny) rozkład prawdopodobieństwa na przedziale $[a,b]$ charakteryzuje się tym, że jego gęstośc na tym przedziale jest stała i różna od zera, a poza wspomnianym przedziałem przyjmuje wartości $0$ jak zostało to przedstawione na rysunku \ref{fig:jednostajny}.

\begin{figure}[H]
\centering
\includegraphics[width=.6\textwidth]{../Obrazy/rownomierny.png} 
\caption{Gęstość rozkładu jednostajnego.}
\label{fig:jednostajny}
\end{figure}

Zaimplementowano i przetestowano 2 rodzaje generatorów: oparty na przebiegu piłokształtnym oraz generator liniowy. 
Wyniki badań zostały przedstawione w dalszej części sprawozdania.

\subsection{Generator oparty na przekształceniu piłokształtnym}
Sygnał piłokształtny zawdzięcza swoją nazwę graficznej reprezentacji, która przypomina zęby piły (rysunek \ref{fig:pila}). Odpowiednie próbkowanie takiego sygnału pozwala na generowanie liczb losowych o rozkładzie jednostajnym.
\begin{figure}[!h]
\centering
\includegraphics[width=.6\textwidth]{../Obrazy/pila.png} 
\caption{Przebieg piłokształtny o 7 ,,zębach'' wygenerowany na przedziale $[0,1)$ w środowisku Python}
\label{fig:pila}
\end{figure}

Generacja liczb losowych z takiego przebiegu opiera się o algorytm, którego istotą jest równanie \ref{eq:pila}. 


\begin{equation}\label{eq:pila}
X_{n+1} = X_{n} \cdot Z - \lfloor X_{n} \cdot Z \rfloor
\end{equation}

\begin{figure}[H]
\centering
\begin{subfigure}{\textwidth}
\includegraphics[width=\textwidth]{../Obrazy/pila1.png} 
\end{subfigure}
\begin{subfigure}{\textwidth}
\includegraphics[width=\textwidth]{../Obrazy/pila2.png} 
\end{subfigure}
\begin{subfigure}{\textwidth}
\includegraphics[width=\textwidth]{../Obrazy/pila3.png} 
\end{subfigure}
\caption{Histogramy generowanych próbek z przebiegu piłokształtnego}
\label{fig:pila1}
\end{figure}

Jak można od razu zauważyć dla liczby zębów piły, czyli $Z=1$ histogram przedstawia 1 słupek.
Dzieje się tak, gdyż równanie \ref{eq:pila} przyjmuje w takim wypadku postać \ref{eq:pilaZ1} co oznacza, że każdy kolejny element jest powtórzeniem poprzedniego.
Dodatkowo dla parzystej liczby zębów również występuje problem skupienia zdecydowanej większości liczb w jednym przedziale.
Sprawdzając poszczególne liczby wygenerowane dla takiej ilości ,,zębów'' (rysunek \ref{fig:pilaZ2}) można zauważyć, że liczby generowane są na całym przedziale, lecz po pewnym czasie wszystkie liczby są zerami. Dzieje się tak dlatego, że podczas mnożenia przez 2 w pewnym momencie ,,zerują się'' dalsze miejsca po przecinku co prowadzi do sytuacji, gdzie każda kolejna liczba będzie zerem.


\begin{equation}\label{eq:pilaZ1}
X_{n+1} = X_{n} - \lfloor X_{n}\rfloor
\end{equation}

\begin{figure}[!h]
\centering
\includegraphics[width=.6\textwidth]{../Obrazy/pilaZ2.png} 
\caption{Wykres kolejnych generowanych liczb dla liczby ,,zębów'' piły równej 2}
\label{fig:pilaZ2}
\end{figure}

\subsection{Generator liniowy}
Badając o właściwości generatorów liniowych ma się na myśli generatory zapisywane w postaci przedstawionej jako równanie \ref{eq:liniowy}

\begin{equation}\label{eq:liniowy}
X_{n+1} = (a_{1} \cdot X_{n} + a_{2} \cdot X_{n-1} + a_{3} \cdot X_{n-2} + ... + a_{k} \cdot X_{n-k+1} + c )\ mod\ m
\end{equation}
, gdzie $a_{1} - a_{k},\ c,\ m$ są pewnymi stałymi parametrami generatora, a operacja $mod\ m$ oznacza wyznaczanie reszty z dzieleia przez $m$.

W szczególnym przypadku, gdy $k=0$ to opis generatora przyjmuje postać równania \ref{eq:liniowyk0}.
Co więcej jeżeli $c = 0$ to taki generator nazywany jest generatorem multiplikatywnym, z kolei, gdy $c \neq 0$ -- generatorem mieszanym.
\begin{equation}\label{eq:liniowyk0}
X_{n+1} = (a \cdot X_{n}+ c )\ mod\ m
\end{equation}

Do celów badań zaimplementowano 3 generatory multiplikatywne oraz generator mieszany, a ich parametry przedstawiono w tabeli \ref{tab:parametry}.
Zestawy parametrów zaczerpnięto z książki Wieczorkowskiego i Zielińskiego \cite{generatory1997Wieczorkowski}.
Wygenerowano przy jego pomocy po 1000 liczb dla każdego zestawu parametrów. 
Wyniki przedstawiono w postaci histogramu dzielącego przedział $[0,m)$ na 10 podprzedziałów.
Histogram został przedstawiony na rysunku \ref{fig:histLiniowy}.
Jak widać histogramy przypominają (z niewielkimi odchyleniami) wykres gęstości rozkładu jednostajnego.


\begin{table}[H]
\begin{center}
\begin{tabular}{|c|c|c|c|}
\hline
\textbf{a} & \textbf{c} & \textbf{m} & \textbf{Źródło} \\
\hline
$69069$ & $1$ & $2^{32}$ & Marsaglia (1972) \\
\hline
$40692$ & $0$ & $2^{31}-249$ & L'Ecuyer (1988) \\
\hline
$2^{2} \cdot 23^{7} + 1$ & $0$ & $2^{35}$ & Zielinski (1966) \\
\hline
$1099087573$ & $0$ & $2^{32}$ & Fishman (1990) \\
\hline
\end{tabular}
\caption{Tabela parametrów generatorów liniowych.}
\label{tab:parametry}
\end{center}
\end{table}


\begin{figure}[H]
\centering
\includegraphics[width=\textwidth]{../Obrazy/histLiniowy.png} 
\caption{Histogramy wyników generacji liczb losowych generatora liniowego dla poszczególnych zestawów parametrów.}
\label{fig:histLiniowy}
\end{figure}


\section{Metoda odwrotnej dystrybuanty}
Chcąc wygenerować liczby losowe o dowolnym rozkładzie można wykorzystać algorytm nazywany metodą odwrotnej dystrybuanty.
Jak sama nazwa wskazuje, bazuje ona na wykorzystaniu dystrybuanty rozkładu do generowania liczb.
Podstawowym jednak założeniem jest, że ,,jeżeli istnieje gęstość g rozkładu $\mu$ na $\textbf{R}$, to $	F_{\mu}(t) = \int_{-\infty}^{t}g(x)dx $''\cite{wstep2001jakubowski}. 
Dodatkowo niech U będzie zmienną losową o rozkładzie jednostajnym na $[0,1)$.
Dzięki takim założeniom można wyznaczyć nową zmienną losową $X = F^{-1}(U)$, wtedy zmienna losowa $X$ ma rozkład prawdopodobieństwa o dystrybuancie $F$.
Zatem generując ciąg liczb $X_{i} = F^{-1}(U_{i})$ otrzymuje się ciąg liczb losowych o rozkładzie z dystrybuantą F.


\noindent
W taki sposób zaimplementowano 4 generatory liczb losowych o rozkładach:
\begin{itemize}
\item dowolny,
\item trójkątnym,
\item wykładniczym,
\item Laplace'a.
\end{itemize}


\subsection{Rozkład dowolny}
Dla takiego rozkładu gęstość (przedstawiona wcześniej w postaci wykresu ogólnego na rysunku \ref{eq:liniowy}) można przedstawić jako
\begin{equation}
f(x) = \begin{cases} 2x, & \mbox{dla } x \in [0,1] \\ 0, & \mbox{dla } x \in (-\infty,0) \bigcup (1,\infty) \end{cases}
\end{equation}

\begin{figure}[H]
\centering
\includegraphics[width=\textwidth]{../Obrazy/jednolity.png} 
\caption{Gęstość rozkładu, wykres dystrybuanty oraz jej odwrotności, histogram wygenerowanych liczb oraz wygenerowane liczby z rozkładu równomiernego}
\label{fig:jednorodnyL2}
\end{figure}


\subsection{Rozkład trójkątny}
Gęstość rozkładu prawdopodobieństwa dla implementowanego generatora wynosi:
\begin{equation}\label{eq:trojkatny}
f(x) = \begin{cases} x + 1, & \mbox{dla } x \in (-1,0) \\ x - 1, & \mbox{dla } x \in [0,1) \\ 0, & \mbox{dla } x \not\in (-1,1) \end{cases}
\end{equation}

\begin{figure}[H]
\centering
\includegraphics[width=\textwidth]{../Obrazy/trojkatny.png} 
\caption{Gęstość rozkładu, wykres dystrybuanty oraz jej odwrotności, histogram wygenerowanych liczb oraz wygenerowane liczby z rozkładu trójkątnego}
\label{fig:trojkatnyL2}
\end{figure}

\subsection{Rozkład wykładniczy}
Gęstość wykładniczego rozkładu prawdopodobieństwa:
\begin{equation}
f(x) = e^{-x},\ x \geq 0
\end{equation}

\begin{figure}[H]
\centering
\includegraphics[width=\textwidth]{../Obrazy/wykladniczy.png} 
\caption{Gęstość rozkładu, wykres dystrybuanty oraz jej odwrotności, histogram wygenerowanych liczb oraz wygenerowane liczby z rozkładu wykładniczego}
\label{fig:wykladniczyL2}
\end{figure}

\subsection{Rozkład Laplace'a}
Gęstość rozkładu Laplace'a w ogólnym przypadku przedstawia się jako:
\begin{equation}
f(x) = \frac{1}{2b}\begin{cases} e^{-\frac{\mu-x}{b}}, & \mbox{dla } x < \mu \\ e^{-\frac{x-\mu}{b}}, & \mbox{dla } x \leq \mu\end{cases}
\end{equation}


\begin{figure}[H]
\centering
\includegraphics[width=\textwidth]{../Obrazy/laplace.png} 
\caption{Gęstość rozkładu, wykres dystrybuanty oraz jej odwrotności, histogram wygenerowanych liczb oraz wygenerowane liczby z rozkładu Laplace'a}
\label{fig:laplaceL2}
\end{figure}


\section{Metoda odrzucania}
Algorytm generacji liczb losowych w metodzie odrzucania wiąże się z wykorzystaniem dwóch gęstości rozkładów: $f(x)$ i $g(x)$, czyli kolejno gęstości rozkładu prawdopodobieństwa zmiennej X, której realizacja ma zostać wygenerowana oraz pomocnicza gęstość innej zmiennej losowej.
Dodatkowo wymagane jest istnienie parametru $c>0$  dla którego spełniona jest zależność \ref{eq:c_odrzucania}.

\begin{equation}\label{eq:c_odrzucania}
f(x) \leq cg(x),\ dla\ x\in\textbf{R}
\end{equation}

\subsection{Rozkład trójkątny}
Dla przykładu przyjęto funkcję gęstości rozkładu prawdopodobieństwa z równania \ref{eq:trojkatny}, czyli rozkładu trójkątnego.
Dodatkowo przyjęto funkcję pomocniczą ograniczającą funkcję gęstości rozkładu, z którego miały zostać wygenerowane liczby losowe przyjęto rozkład równomierny w postaci \ref{eq:rownomierny}.
Oczekiwano, że histogram próbek wygenerowanych przy pomocy metody odrzucania, dla tego przypadku będzie przypominał rozkład trójkątny.

\begin{equation}\label{eq:rownomierny}
f(x) = \begin{cases} \frac{1}{b-a}, & \mbox{dla } x \in [a,b] \\ 0 , & \mbox{dla } x \not\in [a,b] \end{cases}
\end{equation}

Na rysunku zostały przedstawione wykresy wynikowe dla generowanych próbek


\section{Wnioski}
\begin{itemize}
\item Na podstawie badań metody odrwotnej dystrybuanty można powiedzieć, że jest metoda dość uniwersalna, pozwalająca generować liczby losowe o dowolnym rozkładzie (przy założeniu, że funkcja gęstości rozkładu prawdopodobieństwa jest całkowalna).
Dla prostych do odwrócenia funkcji nie widać w niej większych wad, jednak gdyby chcieć uzyskać generator z funkcji, której dystrybuanta jest niezwykle trudna do odwrócenia zadanie może się znacząco skomplikować.
\end{itemize}
\bibliographystyle{plabbrv} 

\bibliography{../bibliografia}

\end{document}